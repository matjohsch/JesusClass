
\documentclass[a4paper,12pt]{article}
%%%%%%%%%%%%%%%%%%%%%%%%%%%%%%%%%%%%%%%%%%%%%%%%%%%%%%%%%%%%%%%%%%%%%%%%%%%%%%%%%%%%%%%%%%%%%%%%%%%%%%%%%%%%%%%%%%%%%%%%%%%%%%%%%%%%%%%%%%%%%%%%%%%%%%%%%%%%%%%%%%%%%%%%%%%%%%%%%%%%%%%%%%%%%%%%%%%%%%%%%%%%%%%%%%%%%%%%%%%%%%%%%%%%%%%%%%%%%%%%%%%%%%%%%%%%
\usepackage{amsfonts}
\usepackage{amssymb}
\usepackage{amsmath}
\usepackage{eurosym}
\usepackage{latexsym}
\usepackage{graphicx}
\usepackage{longtable}
\usepackage{portland}
\usepackage{lscape}
\usepackage[onehalfspacing]{setspace}
\usepackage{footmisc}
\usepackage{fancyhdr}
\usepackage{hyphenat}
\usepackage{rotating}
\usepackage[USenglish]{babel}
\usepackage{array}
\usepackage{tabularx}
\usepackage{chicago}
\usepackage{theorem}
\usepackage{multirow}
\usepackage{epstopdf}
\usepackage[left=1in,right=1in,top=1in,bottom=1in]{geometry}

\setcounter{MaxMatrixCols}{10}
%TCIDATA{OutputFilter=Latex.dll}
%TCIDATA{Version=5.00.0.2606}
%TCIDATA{<META NAME="SaveForMode" CONTENT="1">}
%TCIDATA{BibliographyScheme=Manual}
%TCIDATA{LastRevised=Monday, September 09, 2013 21:34:21}
%TCIDATA{<META NAME="GraphicsSave" CONTENT="32">}

\newtheorem{theorem}{Theorem}
\newtheorem{acknowledgement}[theorem]{Acknowledgement}
\newtheorem{algorithm}[theorem]{Algorithm}
\newtheorem{assum}{Assumption}
\newtheorem{axiom}[theorem]{Axiom}
\newtheorem{case}[theorem]{Case}
\newtheorem{claim}[theorem]{Claim}
\newtheorem{conclusion}[theorem]{Conclusion}
\newtheorem{condition}[theorem]{Condition}
\newtheorem{conjecture}[theorem]{Conjecture}
\newtheorem{corollary}[theorem]{Corollary}
\newtheorem{criterion}[theorem]{Criterion}
\newtheorem{definition}{Definition}
\newtheorem{example}[theorem]{Example}
\newtheorem{exercise}[theorem]{Exercise}
\newtheorem{lemma}[theorem]{Lemma}
\newtheorem{notation}[theorem]{Notation}
\newtheorem{problem}[theorem]{Problem}
\newtheorem{proposition}{Proposition}
\newtheorem{remark}[theorem]{Remark}
\newtheorem{solution}[theorem]{Solution}
\newtheorem{summary}[theorem]{Summary}
\newtheorem{observation}{Observation}
\newtheorem{result}{Result}

\begin{document}

\title{Steady State Values of RBC with endogenous labour supply}
\author{Matthias Sch\"{o}n\thanks{%
CMR, University of Cologne; Albertus-Magnus-Platz; 50923 K\"{o}ln; Germany;
E-mail: m.schoen@wiso.uni-koeln.de}}
\date{\today }
\maketitle

\begin{abstract}
This is part of the first assignment for the course ECON 714-001
Quantitative Macro Theory by Jesus Fernandez Villaverde. It contains the
theoretical foundation of the model and and the calculaton of the steady
state values of leisure to calibrate the model. The programming code for the
whole model is provided for Matlab and Fortran. 
\end{abstract}

\newpage \pagenumbering{arabic} \renewcommand{\thefootnote}{%
\arabic{footnote}} \setcounter{footnote}{0}

\section*{Description of the RBC Model}

Consider a simple RBC model consisting of a representative household whose
utility is given by 
\begin{equation*}
U=\sum_{t=1}^{T}\beta ^{t-1}u\left( c_{t},1-l_{t}\right) ,
\end{equation*}%
where $c$ is consumption and $l$ is time spend in production.

The household has a constant returns to scale technology for producing
output given by 
\begin{equation*}
y_{t}=z_{t}k_{t}^{\alpha }l_{t}^{1-\alpha }
\end{equation*}
where $k_{t}$ ist capital and $z_{t}$is a productivity parameter follwing an
AR(1) process. It also faces a budget constraint 
\begin{equation*}
k_{t+1}=\left( 1-\delta \right) k_{t}+y_{t}-c_{t}.
\end{equation*}

The constraints collapses to 
\begin{equation*}
k_{t+1}=\left( 1-\delta \right) k_{t}+z_{t}k_{t}^{\alpha }l_{t}^{1-\alpha
}-c_{t}.
\end{equation*}%
Let $\lambda _{t}$ be the multiplier of this constraint and form 
\begin{equation*}
V\left( k_{t},z_{t}\right) =u\left( c_{t},1-l_{t}\right) +\beta E_{t}V\left(
k_{t+1},z_{t+1}\right) +\lambda _{t}\left( \left( 1-\delta \right)
k_{t}+z_{t}k_{t}^{\alpha }l_{t}^{1-\alpha }-c_{t}-k_{t+1}\right) .
\end{equation*}

The first order condition with respect to $c$ is%
\begin{equation}
\frac{\partial u}{\partial c_{t}}-\lambda _{t}\left( -1\right)
=0\Leftrightarrow \frac{\partial u}{\partial c}=\lambda _{t}  \label{I}
\end{equation}%
w.r.t. $l$ is%
\begin{equation}
\frac{\partial u}{\partial l_{t}}+\lambda _{t}z_{t}\left( 1-\alpha \right)
k_{t}^{\alpha }l_{t}^{-\alpha }=0  \label{II}
\end{equation}%
w.r.t. $k_{t+1}$is 
\begin{equation}
\beta E_{t}\frac{\partial V}{\partial k}\left( k_{t+1},z_{t+1}\right)
+\lambda _{t}\left( -1\right) =0\Leftrightarrow \lambda _{t}=\beta E_{t}%
\frac{\partial V}{\partial k}\left( k_{t+1},z_{t+1}\right) .  \label{III}
\end{equation}%
The envelope condition is 
\begin{equation}
\frac{\partial V}{\partial k}\left( k_{t},z_{t}\right) =\lambda _{t}\left(
\left( 1-\delta \right) +z_{t}\alpha k_{t}^{\alpha -1}l_{t}^{1-\alpha
}\right) .  \label{IV}
\end{equation}

If we combine (\ref{I}) and (\ref{II}) yields%
\begin{equation}
-\frac{\partial u}{\partial l_{t}}=\frac{\partial u}{\partial c}z_{t}\left(
1-\alpha \right) k_{t}^{\alpha }l_{t}^{-\alpha }  \label{V}
\end{equation}%
and (\ref{I}) and (\ref{III}) lead to%
\begin{equation}
\frac{\partial u}{\partial c}=\beta E_{t}\frac{\partial V}{\partial k}\left(
k_{t+1},z_{t+1}\right) .  \label{VI}
\end{equation}%
Shifting (\ref{IV}) forward one period in time and taking expectations gives%
\begin{equation}
E_{t}\frac{\partial V}{\partial k}\left( k_{t+1},z_{t+1}\right)
=E_{t}\lambda _{t+1}\left( \left( 1-\delta \right) +z_{t+1}\alpha
k_{t+1}^{\alpha -1}l_{t+1}^{1-\alpha }\right)   \label{VII}
\end{equation}

Now let assume that $u\left( c_{t},1-l_{t}\right) =\log (c_{t})$-$\psi \frac{%
l^{2}}{2}$ and $f(k_{t},l_{t})=k^{\alpha }l^{1-\alpha }$. Equation (\ref{VI}%
) becomes 
\begin{equation}
\frac{1}{c_{t}}=\beta E_{t}\frac{\partial V}{\partial k}\left(
k_{t+1},z_{t+1}\right)   \label{VIII}
\end{equation}%
equation (\ref{V}) becomes 
\begin{equation*}
\psi l_{t}=\frac{1}{c_{t}}z_{t}\left( 1-\alpha \right) k_{t}^{\alpha
}l_{t}^{-\alpha }
\end{equation*}%
and (\ref{VII}) 
\begin{equation}
E_{t}\frac{\partial V}{\partial k}\left( k_{t+1},z_{t+1}\right) =E_{t}\frac{1%
}{c_{t+1}}\left( \left( 1-\delta \right) +z_{t+1}\alpha k_{t+1}^{\alpha
-1}l_{t+1}^{1-\alpha }\right)   \label{IX}
\end{equation}%
Combing (\ref{VIII}) and (\ref{IX})yields%
\begin{equation*}
\frac{1}{c_{t}}=\beta E_{t}\frac{1}{c_{t+1}}\left( \left( 1-\delta \right)
+z_{t+1}\alpha k_{t+1}^{\alpha -1}l_{t+1}^{1-\alpha }\right) 
\end{equation*}

In the nonstochastic steady state holds $c=c_{t}=c_{t+1}$, $k=k_{t}=k_{t+1}$%
, $l=l_{t}=l_{t+1}$ and $z_{t}=z_{t+1}=1$. So we can write the steady state
equations as  
\begin{eqnarray}
\frac{1}{c} &=&\frac{1}{c}\beta \left( \left( 1-\delta \right) +\alpha
k^{\alpha -1}l^{1-\alpha }\right)   \notag \\
k &=&\left( \frac{\alpha \beta }{1-\beta +\beta \delta }\right) ^{\frac{1}{%
1-\alpha }}l\Leftrightarrow \frac{k}{l}=\left( \frac{\alpha \beta }{1-\beta
+\beta \delta }\right) ^{\frac{1}{1-\alpha }}  \label{X}
\end{eqnarray}%
and 
\begin{equation}
c=\frac{\left( 1-\alpha \right) }{\psi }\left( \frac{k}{l}\right) ^{\alpha
}l^{-1}=\frac{\left( 1-\alpha \right) }{\psi }\left( \frac{\alpha \beta }{%
1-\beta +\beta \delta }\right) ^{\frac{\alpha }{1-\alpha }}l^{-1}  \label{XI}
\end{equation}%
By pluging \ref{X} and \ref{XI} in the budget constraint 
\begin{equation*}
\delta \left( \frac{\alpha \beta }{1-\beta +\beta \delta }\right) ^{\frac{1}{%
1-\alpha }}l=\left( \left( \frac{\alpha \beta }{1-\beta +\beta \delta }%
\right) ^{\frac{1}{1-\alpha }}l\right) ^{\alpha }l^{1-\alpha }-\frac{\left(
1-\alpha \right) }{\psi }\left( \frac{k}{l}\right) ^{\alpha }l^{-1}
\end{equation*}%
we can now compute the steady state values of labour, capital, consumption
and output.%
\begin{eqnarray*}
l_{ss} &=&\left[ \frac{\left( 1-\alpha \right) }{\psi }\frac{1}{\left(
1-\delta \frac{\alpha \beta }{1-\beta +\beta \delta }\right) }\right] ^{1/2}
\\
k_{ss} &=&\left( \frac{\alpha \beta }{1-\beta +\beta \delta }\right) ^{\frac{%
1}{1-\alpha }}l_{ss} \\
c_{ss} &=&\frac{\left( 1-\alpha \right) }{\psi }\left( \frac{\alpha \beta }{%
1-\beta +\beta \delta }\right) ^{\frac{\alpha }{1-\alpha }}l_{ss}^{-1} \\
y_{ss} &=&\left( \frac{\alpha \beta }{1-\beta +\beta \delta }\right) ^{\frac{%
\alpha }{1-\alpha }}l_{ss}
\end{eqnarray*}
As we want to have a steady state labour supply of 1/3 we have to calibrate
the utility parameter $\psi $ accordingly. By rearranging the steady state
equation for labour we get%
\begin{equation*}
\psi =\frac{\left( 1-\alpha \right) }{l_{ss}^{2}}\frac{1}{\left( 1-\delta 
\frac{\alpha \beta }{1-\beta +\beta \delta }\right) }.
\end{equation*}%
Plugging in the parameter values and and $l_{ss}=1/3$ yields 
\begin{equation*}
\psi =7.5981.
\end{equation*}%
The corresponding other steady state values are 
\begin{eqnarray*}
k_{ss} &=&1.1909 \\
c_{ss} &=&0.4024 \\
y_{ss} &=&0.5096.
\end{eqnarray*}

\end{document}
