
\documentclass[a4paper,12pt]{article}
%%%%%%%%%%%%%%%%%%%%%%%%%%%%%%%%%%%%%%%%%%%%%%%%%%%%%%%%%%%%%%%%%%%%%%%%%%%%%%%%%%%%%%%%%%%%%%%%%%%%%%%%%%%%%%%%%%%%%%%%%%%%%%%%%%%%%%%%%%%%%%%%%%%%%%%%%%%%%%%%%%%%%%%%%%%%%%%%%%%%%%%%%%%%%%%%%%%%%%%%%%%%%%%%%%%%%%%%%%%%%%%%%%%%%%%%%%%%%%%%%%%%%%%%%%%%
\usepackage{amsfonts}
\usepackage{amssymb}
\usepackage{amsmath}
\usepackage{eurosym}
\usepackage{latexsym}
\usepackage{graphicx}
\usepackage{longtable}
\usepackage{portland}
\usepackage{lscape}
\usepackage[onehalfspacing]{setspace}
\usepackage{footmisc}
\usepackage{fancyhdr}
\usepackage{hyphenat}
\usepackage{rotating}
\usepackage[USenglish]{babel}
\usepackage{array}
\usepackage{tabularx}
\usepackage{chicago}
\usepackage{theorem}
\usepackage{multirow}
\usepackage{epstopdf}
\usepackage[left=1in,right=1in,top=1in,bottom=1in]{geometry}

\setcounter{MaxMatrixCols}{10}
%TCIDATA{OutputFilter=Latex.dll}
%TCIDATA{Version=5.00.0.2606}
%TCIDATA{<META NAME="SaveForMode" CONTENT="1">}
%TCIDATA{BibliographyScheme=Manual}
%TCIDATA{LastRevised=Saturday, September 07, 2013 20:04:57}
%TCIDATA{<META NAME="GraphicsSave" CONTENT="32">}

\newtheorem{theorem}{Theorem}
\newtheorem{acknowledgement}[theorem]{Acknowledgement}
\newtheorem{algorithm}[theorem]{Algorithm}
\newtheorem{assum}{Assumption}
\newtheorem{axiom}[theorem]{Axiom}
\newtheorem{case}[theorem]{Case}
\newtheorem{claim}[theorem]{Claim}
\newtheorem{conclusion}[theorem]{Conclusion}
\newtheorem{condition}[theorem]{Condition}
\newtheorem{conjecture}[theorem]{Conjecture}
\newtheorem{corollary}[theorem]{Corollary}
\newtheorem{criterion}[theorem]{Criterion}
\newtheorem{definition}{Definition}
\newtheorem{example}[theorem]{Example}
\newtheorem{exercise}[theorem]{Exercise}
\newtheorem{lemma}[theorem]{Lemma}
\newtheorem{notation}[theorem]{Notation}
\newtheorem{problem}[theorem]{Problem}
\newtheorem{proposition}{Proposition}
\newtheorem{remark}[theorem]{Remark}
\newtheorem{solution}[theorem]{Solution}
\newtheorem{summary}[theorem]{Summary}
\newtheorem{observation}{Observation}
\newtheorem{result}{Result}
\input{tcilatex}

\begin{document}

\title{REal Business Cycle with endogenous labour choice}
\author{Matthias Sch\"{o}n\thanks{%
CMR, University of Cologne; Albertus-Magnus-Platz; 50923 K\"{o}ln; Germany;
E-mail: m.schoen@wiso.uni-koeln.de}}
\date{\today }
\maketitle

\begin{abstract}
This is documetation to the homework for the course quantitative Macro
Theory by Jesus Fernandez Villaverde. It contains the theoretical foundation
of the model and and a short description of the programs that are also
provided. I choose to code in Matlab and Fortran.
\end{abstract}

\newpage \pagenumbering{arabic} \renewcommand{\thefootnote}{%
\arabic{footnote}} \setcounter{footnote}{0}

\section*{Description of the RBC Model}

Consider a simple RBC model consisting of a representative household whose
utility is given by 
\begin{equation*}
U=\sum_{t=1}^{T}\beta ^{t-1}u\left( c_{t},1-l_{t}\right) ,
\end{equation*}%
where c is consumption and l is time spend in production. 

The household has a constant returns to scale technology for producing
output given by 
\begin{equation*}
y_{t}=z_{t}k_{t}^{\alpha }l_{t}^{1-\alpha }
\end{equation*}
where $k_{t}$ ist capital and $z_{t}$is a productivity parameter follwing an
AR(1) process. It also faces a budget constraint 
\begin{equation*}
k_{t+1}=\left( 1-\delta \right) k_{t}+y_{t}-c_{t}.
\end{equation*}

The constraints collapses to 
\begin{equation*}
k_{t+1}=\left( 1-\delta \right) k_{t}+z_{t}k_{t}^{\alpha }l_{t}^{1-\alpha
}-c_{t}.
\end{equation*}%
Let $\lambda _{t}$ be the multiplier of this constraint and form 
\begin{equation*}
V\left( k_{t},z_{t}\right) =u\left( c_{t},1-l_{t}\right) +\beta E_{t}V\left(
k_{t+1},z_{t+1}\right) +\lambda _{t}\left( \left( 1-\delta \right)
k_{t}+z_{t}k_{t}^{\alpha }l_{t}^{1-\alpha }-c_{t}-k_{t+1}\right) .
\end{equation*}%
The first order condition for c is%
\begin{equation}
\frac{\partial u}{\partial c_{t}}-\lambda _{t}\left( -1\right)
=0\Leftrightarrow \frac{\partial u}{\partial c}=\lambda _{t}  \tag{I}
\label{I}
\end{equation}%
The first order condition for l is%
\begin{equation}
\frac{\partial u}{\partial l_{t}}+\lambda _{t}z_{t}\left( 1-\alpha \right)
k_{t}^{\alpha }l_{t}^{-\alpha }=0  \tag{II}  \label{II}
\end{equation}%
The first order condition for $k_{t+1}$is 
\begin{equation}
\beta E_{t}\frac{\partial V}{\partial k}\left( k_{t+1},z_{t+1}\right)
+\lambda _{t}\left( -1\right) =0\Leftrightarrow \lambda _{t}=\beta E_{t}%
\frac{\partial V}{\partial k}\left( k_{t+1},z_{t+1}\right) .  \tag{III}
\label{III}
\end{equation}%
The envelope condition is 
\begin{equation}
\frac{\partial V}{\partial k}\left( k_{t},z_{t}\right) =\lambda _{t}\left(
\left( 1-\delta \right) +z_{t}\alpha k_{t}^{\alpha -1}l_{t}^{1-\alpha
}\right)   \tag{IV}  \label{IV}
\end{equation}

If we combine \ref{I} and \ref{II} yields%
\begin{equation}
-\frac{\partial u}{\partial l_{t}}=\frac{\partial u}{\partial c}z_{t}\left(
1-\alpha \right) k_{t}^{\alpha }l_{t}^{-\alpha }  \tag{V}  \label{V}
\end{equation}%
Combine \ref{I} and \ref{III} to get%
\begin{equation}
\frac{\partial u}{\partial c}=\beta E_{t}\frac{\partial V}{\partial k}\left(
k_{t+1},z_{t+1}\right)   \tag{VI}  \label{VI}
\end{equation}%
Shitfing \ref{IV} forward one period in time and taking expectations gives%
\begin{equation}
E_{t}\frac{\partial V}{\partial k}\left( k_{t+1},z_{t+1}\right)
=E_{t}\lambda _{t+1}\left( \left( 1-\delta \right) +z_{t+1}\alpha
k_{t+1}^{\alpha -1}l_{t+1}^{1-\alpha }\right)   \tag{VII}  \label{VII}
\end{equation}

Now let assume that $u\left( c_{t},1-l_{t}\right) =\log (c_{t})$-$\psi \frac{%
l^{2}}{2}$ and $f(k_{t},l_{t})=k^{\alpha }l^{1-\alpha }$. Equation \ref{VI}
becomes  
\begin{equation*}
\frac{1}{c_{t}}=\beta E_{t}\frac{\partial V}{\partial k}\left(
k_{t+1},z_{t+1}\right) 
\end{equation*}

Equation \ref{V} becomes 
\begin{equation*}
\psi l_{t}=\frac{1}{c_{t}}z_{t}\left( 1-\alpha \right) k_{t}^{\alpha
}l_{t}^{-\alpha }
\end{equation*}%
and \ref{VII} 
\begin{equation*}
E_{t}\frac{\partial V}{\partial k}\left( k_{t+1},z_{t+1}\right) =E_{t}\frac{1%
}{c_{t+1}}\left( \left( 1-\delta \right) +z_{t+1}\alpha k_{t+1}^{\alpha
-1}l_{t+1}^{1-\alpha }\right) 
\end{equation*}%
Combing them yields%
\begin{equation*}
\frac{1}{c_{t}}=\beta E_{t}\frac{1}{c_{t+1}}\left( \left( 1-\delta \right)
+z_{t+1}\alpha k_{t+1}^{\alpha -1}l_{t+1}^{1-\alpha }\right) 
\end{equation*}

In the nonstochastic steady state holds $c=c_{t}=c_{t+1}$, $k=k_{t}=k_{t+1}$%
, $l=l_{t}=l_{t+1}$ and $z_{t+1}=z_{t+1}=1.$ 
\begin{eqnarray*}
\frac{1}{c} &=&\frac{1}{c}\beta \left( \left( 1-\delta \right) +\alpha
k^{\alpha -1}l^{1-\alpha }\right)  \\
k &=&\left( \frac{\alpha \beta }{1-\beta +\beta \delta }\right) ^{\frac{1}{%
1-\alpha }}l\Leftrightarrow \frac{k}{l}=\left( \frac{\alpha \beta }{1-\beta
+\beta \delta }\right) ^{\frac{1}{1-\alpha }}
\end{eqnarray*}%
and 
\begin{equation*}
c=\frac{\left( 1-\alpha \right) }{\psi }\left( \frac{k}{l}\right) ^{\alpha
}l^{-1}=\frac{\left( 1-\alpha \right) }{\psi }\left( \frac{\alpha \beta }{%
1-\beta +\beta \delta }\right) ^{\frac{\alpha }{1-\alpha }}l^{-1}
\end{equation*}%
Pluging in this in the budget constraint we can compute the steady state
values of capital labour and consumption.%
\begin{eqnarray*}
k &=&\left( 1-\delta \right) k+k^{\alpha }l^{1-\alpha }-c \\
\delta k &=&k^{\alpha }l^{1-\alpha }-c \\
\delta \left( \frac{\alpha \beta }{1-\beta +\beta \delta }\right) ^{\frac{1}{%
1-\alpha }}l &=&\left( \left( \frac{\alpha \beta }{1-\beta +\beta \delta }%
\right) ^{\frac{1}{1-\alpha }}l\right) ^{\alpha }l^{1-\alpha }-\frac{\left(
1-\alpha \right) }{\psi }\left( \frac{k}{l}\right) ^{\alpha }l^{-1} \\
l_{ss} &=&\left( \frac{\left( 1-\alpha \right) }{\psi }\frac{1}{\left(
1-\delta \frac{\alpha \beta }{1-\beta +\beta \delta }\right) }\right) ^{1/2}
\\
k_{ss} &=&\left( \frac{\alpha \beta }{1-\beta +\beta \delta }\right) ^{\frac{%
1}{1-\alpha }}\left( \frac{\left( 1-\alpha \right) }{\psi }\frac{1}{\left(
1-\delta \frac{\alpha \beta }{1-\beta +\beta \delta }\right) }\right) ^{1/2}
\\
c &=&\frac{\left( 1-\alpha \right) }{\psi }\left( \frac{\alpha \beta }{%
1-\beta +\beta \delta }\right) ^{\frac{\alpha }{1-\alpha }}\left( \frac{%
\left( 1-\alpha \right) }{\psi }\frac{1}{\left( 1-\delta \frac{\alpha \beta 
}{1-\beta +\beta \delta }\right) }\right) ^{-1/2}
\end{eqnarray*}%
For $\psi $ that leads to l$_{ss}$ is%
\begin{eqnarray*}
\psi  &=&\frac{\left( 1-\alpha \right) }{l_{ss}^{2}}\frac{1}{\left( 1-\delta 
\frac{\alpha \beta }{1-\beta +\beta \delta }\right) }=\frac{\left( 1-\frac{1%
}{3}\right) }{\left( \frac{1}{3}\right) ^{2}}\frac{1}{\left( 1-\frac{9}{100}%
\frac{\frac{1}{3}\frac{19}{20}}{1-\frac{19}{20}+\frac{19}{20}\frac{9}{100}}%
\right) } \\
&=&\frac{\frac{2}{3}}{\frac{1}{9}}\frac{1}{\left( 1-\frac{\frac{171}{6000}}{%
\frac{271}{2000}}\right) }==6\frac{1}{\left( \frac{214}{271}\right) }=6\frac{%
271}{214}=\frac{3\ast 271}{107}=\frac{813}{107}=7.598
\end{eqnarray*}

\end{document}
